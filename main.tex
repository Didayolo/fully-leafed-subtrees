\documentclass{article}
\usepackage[utf8]{inputenc}

\title{Prise de notes au Lacim}
\author{Adrien Pavao}
\date{Juin 2017}

\begin{document}

\maketitle

\section{Introduction}

\section{Problèmes à étudier}

\subsection{Conjecture 4/3} 

Les sous-mots antipalindromiques.

Résumé de la conjecture

\textbf{Antipalindrome}
Qu'est-ce qu'un antipalindrome

Ref: http://ojs.statsbiblioteket.dk/index.php/brics/article/download/20073/17699
Ref: https://cermics.enpc.fr/~meuniefr/OpenProblems.html

\subsection{Codes couvrants} 

(les antennes et les téléphones)
Résumé du problème

Ils ont déjà été étudiés dans les grilles carrées, hexagonales et triangulaires, mais pas encore dans la grille du roi.
 
\subsubseciogrille carrée

Ref: http://www.sciencedirect.com/science/article/pii/S0012365X02007446
%http://www.discuss.wmie.uz.zgora.pl/php/discuss3.php?ip=&url=bwww_praca&nId=2454&nIdCzasopisma=402&nIdSesji=-1
%http://math.nsc.ru/~puzynina/3_pc3.ps
% - dans d'autres grilles
%http://link.springer.com/article/10.1134/S0037446606010101

\subsection{Sous-arbres induits}

La question est "Quel est le nombre maximal de feuilles réalisé par un sous-arbre induit à n sommets dans un graphe donné?"
%Le document en pièce jointe contient un résultat de complexité et algorithme polynomial pour les arbres.
\\
Peut-on obtenir un algorithme polynomial pour les graphes de largeur arborescente fixée ?
%document piece jointe fully tanani

Quelques définitions :

\begin{itemize}
\item \textbf{Arbre}
Un arbre est un graphe non orienté dans lequel deux sommets quelconques sont reliés par un unique chemin (acyclique). Une feuille est un sommet de degré 1 dans un arbre.

\item \textbf{Largeur arborescente}
La largeur d'arbre est une façon de savoir si, intuitivement un graphe est proche d'un arbre.

\item \textbf{Sous-arbre induit}
Un sous-graphe induit est un sous-graphe défini par un sous ensemble de sommets. 
\\
Formellement, H est un sous-graphe induit de G si, pour tout couple (x, y) de sommets de H, x est connecté à y dans H si et seulement si x est connecté à y dans G. Autre formulation de la condition : l'ensemble des arêtes de H correspond à l'ensemble des arêtes de G incidentes à deux sommets de H.
\\
Vulgairement, on choisit les sommets du sous-graphe et on met les arêtes telles qu'elles étaient dans le graphe d'origine, sauf celles à qui il manque un sommet bien-sûr. 
\\
Un sous-arbre induit est un sous-graphe induit qui est un arbre.

\item \textbf{Spanning tree}
Un spanning tree d'un graphe à n sommets est un sous-ensemble de n-1 arrêtes qui forme un arbre.

\item \textbf{Leaf number}
Nombre maximal de feuilles d'un spanning tree d'un graphe.

\end{itemize}

Traduction du problème dans un SAT solver ?
Etudier graphes avec Sage ?
%Complexité et largeur arborescente

% polyminoes
% document piece jointe iwoca

\end{document}

